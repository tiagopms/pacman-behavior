%TCIDATA{LaTeXparent=0,0,relatorio.tex}
 


\chapter{Introdução} \label{Chap:Intro}

% Resumo opcional. Comentar se não usar.
\resumodocapitulo{ ``Opcional, geralmente se colocam pequenos resumos
ou citações interessantes.'' -- Murilo M. Marinho}

Esse capítulo faz uma introdução do leitor ao contexto do trabalho,
define a problemática a ser trabalhada no projeto e resume os resultados
e o manuscrito.


\section{Contextualização}

O câncer é uma doença devastadora e um problema de saúde pública que
afeta milhares de pessoas em todos os países do mundo no contexto
atual de globalização. No mundo, o câncer é responsável por mais de
seis milhões de óbitos a cada ano. 



\section{Definição do problema}

O problema que o trabalho pretende solucionar.


\section{Objetivos do projeto}

Quais os principais objetivos do projeto.


\section{Resultados obtidos}

Os principais resultados obtidos.


\section{Apresentação do manuscrito}

Breve resumo do manuscrito, sobre o quê cada capítulo falará posteriormente.
