%TCIDATA{LaTeXparent=0,0,relatorio.tex}
 


\chapter{Introdução} \label{Chap:Intro}

% Resumo opcional. Comentar se não usar.
\resumodocapitulo{``Do what you can, with what you have, where you are.'' -- Theodore Roosevelt}

\section{Contextualização}

Uma pergunta que nasceu junto com a própria consciência humana e que ainda não possui uma resposta concreta é proposta por Carla Koike como a pergunta universal em sua tese de Doutorado \cite{Koike:2005} ``What can I do next, knowing what I have seen and what I have done?'', ou, em português, ``O que posso fazer a seguir, sabendo o que já vi e o que fiz?''.

Outra questão complexa é definir o que é um robô? Isso é difícil pois essa palavra é aplicada a uma diversidade de máquinas que tem poucos pontos em comum. O dicionário Webster, por exemplo, para levar em conta a diversidade de aplicações dessa palavra, apresenta uma definição tripla pra esse termo, sendo uma delas a seguinte: “um mecanismo guiado por um controle autônomo” \cite{Fievet_2002}. A definição deixa clara essa que é das principais características de um sistema robótico, a autonomia.

Tendo esse conceito e voltando agora a pergunta anterior, modificando-a ligeiramente para “O que devo fazer a seguir, sabendo o que já vi e o que fiz?”, vê-se porque ela é uma das mais feitas no campo da robótica. Quando ela é respondida, ela define o controle autônomo do robô.

Em todas as áreas da robótica muitos esforços tem sido aplicados para se obter uma resposta a essa questão e estratégias diferentes são utilizadas em aplicações diferentes. Para alguns braços robóticos, por exemplo, devido a sua precisão de movimento, o planejamento de movimento é feito deterministicamente. Outras aplicações devem levar em conta as limitações dos atuadores e sensores.

Robótica probabilística (\textit{probabilistic robotics}) é uma área da robótica que leva em conta a incerteza das percepções (sensores) e do controle (atuadores) de um robô. Baseada no campo de estatística, essa estratégia provê o robô com algoritmos mais robustos para lidar com situações do mundo real \cite{Thrun:2005:PR:1121596}. Esse nível de robustez é essencial quando se tem como plataforma um robô móvel, onde sensores e atuadores apresentam incertezas que, se não tratadas, levam a performances menores que as esperadas.

Trazendo essa pergunta para o campo de robótica probabilística obtém-se uma equação que a resume $ P \left( M^t \mid z^{0: t} m^{0: t -1} \pi \right) $, ou seja, qual a probabilidade de se executar uma ação $ m \in M $ no tempo $ t $, sendo que foram observados $ z^{0:t} $ nos tempos de $ 0 $ a $ t $ e foram realizadas as ações $ m^{0:t-1} $ nos tempos $ 0 $ a $ t-1 $. $ \pi $ representa o conhecimento inerente que se tem dado do mundo.

Levando esse problema a um novo nível, pode-se perguntar como fazer um robô aprender dando-se recompensas a ele por certas ações e com isso aumentando sua eficiência com o passar do tempo.


\section{Definição do problema}

Neste trabalho é abordado o problema de planejamento de tarefas para um robô móvel, utilizando aprendizagem por reforço e redes bayesianas. Este problema envolve a escolha de um comportamento e uma ação para serem tomados pelo robô a cada momento tendo como base os dados dos sensores e as ações anteriores tomadas pelo robô, além das recompensas recebidas pelo robô para cada dessas ações executadas. Para isso é preciso integrar os algoritmos (seleção de comportamento e rede bayesiana) sem comprometer a precisão matemática de ambos, utilizando a seleção de comportamento como uma tomada de decisão a alto nível e a rede bayesiana para decidir que ação de baixo nível tomar.


\section{Objetivos do projeto}

O objetivo principal desse trabalho é obter um algoritmo de tomada de decisões hierárquico (seleção de comportamento a alto nível e rede bayesiana a baixo) que tenha alto desempenho num sistema de robótica móvel. Para isso é necessário a integração entre ambos os algoritmos utilizando-se métodos matemáticos confiáveis e a implementação desses algoritmos numa plataforma, com erros sensoriais e de atuação, que será utilizada para testar o sistema.


\section{Resultados obtidos}

O agente conseguiu aprender a escolher os comportamentos satifatóriamente, progredindo ao longo de sua execução. Ele conseguiu também detectar particularidades do sistema onde era executado, tendo sua aprendizagem condicionada ao ambiente em que é utilizado.


\section{Apresentação do manuscrito}

Esse trabalho foi dividido em 5 capítulos: Introdução, Fundamentação Teórica, Desenvolvimento, Análise de Dados e Conclusão.

\begin{itemize}
	\item Introdução --- O capítulo atual, que contextualiza o trabalho e apresenta o problema;
	\item Fundamentação Teórica --- Introduz as teorias matemáticas em que esse trabalho se baseia;
	\item Desenvolvimento --- Mostra a modelagem matemática desenvolvida e utilizada nesse trabalho. Também descreve os testes realidos;
	\item Análise de Dados --- Contém os resultados dos testes realizados e sua análise;
	\item Conclusão --- Conclui o trabalho e apresenta perspectivas de trabalhos futuros que podem ser realizados para expandir a teoria descrita aqui.
\end{itemize}
