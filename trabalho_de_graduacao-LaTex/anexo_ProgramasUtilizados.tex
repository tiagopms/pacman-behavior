%TCIDATA{LaTeXparent=0,0,relatorio.tex}



\chapter{Programas utilizados}

Esse tipo de anexo é puramente opcional e recomendo bastante.

\label{AnFRHINO} 

Reunindo todas as referências obtidas, com um agradecimento especial
à Mariana C. Bernardes, alguns dos programas utilizados serão citados.
\begin{itemize}
\item Escrita do relatório:

\begin{itemize}
\item Usou-se o Lyx como ambiente de desenvolvimento para criar o texto
usando \TeX{}. Ele permite que se use o \TeX{} sem ter que se preocupar
demais com coisas pequenas, como gerenciamento de rótulos e facilita,
infinitamente, a criação de tabelas, matrizes, figuras etc.
\item Usou-se o inkscape para desenhar quadros e para adicionar setas e
textos nas figuras. A capacidade de exportar em .pdf com texto em
\TeX{} é fantástica.
\item Usou-se o Visio para desenho da maioria dos diagramas de blocos. Quando
símbolos \TeX{} foram necessários, exportou-se em .svg e adicionaram-se
os textos no inkscape. Em um dos blocos, usou-se o simulink.
\item Para gerenciamento da bibliografia, usou-se o JabRef. Ele também facilita
a organização da sua biblioteca de referências.
\end{itemize}
\item Simulação:

\begin{itemize}
\item Basicamente se usou o Matlab. O \textit{script }pode ser escrito para
que ele já imprima no .pdf. Assim, se algo for alterado na simulação,
basta apenas executá-la com esses comandos, que o relatório automaticamente
atualiza com as figuras corretas.
\end{itemize}
\item Ambiente de programação:

\begin{itemize}
\item Linux: No Linux escreveu-se usando o gedit e compilação com makefiles.
Não se podia arriscar instalar outro programa e impedir o funcionamento
correto do Xenomai.
\item Windows: Visual Studio. O debbuger ajuda muito no desenvolvimento,
principalmente quando se está trabalhando com alocação dinâmica.
\end{itemize}
\item Algumas bibliotecas específicas

\begin{itemize}
\item Matrix: Para operações com matrizes, usou-se a biblioteca gMatrix
do Professor Geovany A. Borges. A biblioteca já tem diversas funções
prontas e é fácil implementar sobre ela quando necessário.
\item Impressão para leitura no Matlab: gDataLogger do Professor Geovany
A. Borges, em geral. Essa biblioteca deixa transparente a maior parte
do processo, e facilita muito. No caso do simulador G3D, não se conseguiu
usar essa biblioteca, então outra forma foi encontrada. Não é complicado,
apenas realizar um fwrite() com o dado desejado e depois realizar
a leitura.\end{itemize}
\end{itemize}

