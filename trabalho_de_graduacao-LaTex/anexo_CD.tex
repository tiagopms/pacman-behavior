%TCIDATA{LaTeXparent=0,0,relatorio.tex}



\chapter{Descrição do conteúdo do CD}

\label{AnCD} 

O CD segue a seguinte estrutura:
\begin{itemize}
\item leiame.pdf : Essa descrição do CD
\item \textbackslash{}template\_relatorio : Um template de Relatórios de
Graduação, já ajustado e configurado para uso no Lyx.
\item \textbackslash{}Rhino

\begin{itemize}
\item \textbackslash{}codigo : Pasta com uma versão comprimida do software
criado para controle do $\rhino$.
\item \textbackslash{}videos :

\begin{itemize}
\item homing.mov : Video do $\rhino$ executando o procedimento de \textit{homing}.
\end{itemize}
\item \textbackslash{}referencias: diversos manuais de fabricantes que possuem
informações relevantes usadas no desenvolvimento da plataforma.
\item \textbackslash{}matlab: \textit{script} do Matlab que lê os dados
obtidos no experimento com o $\rhino$ e cria os gráficos, para verificação.
\end{itemize}
\item Schunk

\begin{itemize}
\item \textbackslash{}codigo:

\begin{itemize}
\item \textbackslash{}simulacao modo insercao : código desenvolvido na linguagem
C utilizado para realizar a simulação da tarefa de controle no modo
de inserção.
\item \textbackslash{}simulacao modo posicionamento: código desenvolvido,
incluindo o simulador 3D, que realiza a simulação da tarefa de controle
no modo de controle de posição por sensor de força. O simulador também
realiza a outra tarefa de controle se configurado corretamente.
\end{itemize}
\item \textbackslash{}videos:

\begin{itemize}
\item posicionamento.avi : Vídeo do simulador 3D executando o controle de
posicionamento.
\end{itemize}
\item \textbackslash{}referencias: diversos manuais de fabricantes que possuem
informações relevantes.
\item \textbackslash{}modelagem: contém um arquivo .pdf com os parâmetros
$\dh$ obtidos nesse trabalho.
\item \textbackslash{}matlab: 

\begin{itemize}
\item \textbackslash{}simulacao modo insercao: a própria simulação no modo
de inserção.
\item \textbackslash{}simulacao modo posicionamento: \textit{script} do
Matlab que lê os dados obtidos no experimento de posicionamento e
cria os gráficos.\end{itemize}
\end{itemize}
\end{itemize}

