%TCIDATA{LaTeXparent=0,0,relatorio.tex}


\chapter{Conclusões} \label{Chap:Conclusoes}

O algoritmo teve o desempenho desejado e conseguiu aprender como utilizar os diferentes parâmetros para fazer boas escolhas num sistema estocástico. Ele apresentou, ainda, curvas  de aprendizagem diferentes para mapas diferentes, como é esperado.

Os testes mostram, também, que esse algoritmo consegue aprender características específicas dos mapas. Um exemplo é o fato de perceber que um ou outro comportamento, ao ser invocado em uma parte específica do mapa, consegue ganhos que não são inerentes a ele.

O algoritmo, no entanto, possui algumas limitações, como, por exemplo, sua dificuldade para sair de um máximo local, para certas circunstâncias. Outra limitação é ter de se escolher os parâmetros manualmente, de forma que eles consigam descrever bem o sistema, tendo valores linearmente separáveis para diferentes comportamentos.

\section{Perspectivas Futuras}

Uma proposta que pode ser feita para melhorar esse trabalho é a utilização de uma rede neural no lugar da função de ganho atualmente utilizada. Com uma rede neural os parâmetros não teriam de ser escolhidos manualmente e relações mais complexas poderiam ser descobertas, a partir do espectro de probabilidades atual do sistema $ a \in A $.

Outro trabalho que pode ser feito é a criação de uma plataforma física em que esse sistema poderia ser testado fora de simulação, dando suporte para os robôs não se danificarem durante o processo de aprendizagem.
